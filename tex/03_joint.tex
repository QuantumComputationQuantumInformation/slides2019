\documentclass{beamer}
%\usepackage{xspace}
\usepackage{amsmath,amssymb}
\usepackage{graphicx}
%\usepackage{svg}
%\usepackage{pgfpages}
%\pgfpagesuselayout{4 on 1}[a4paper,border shrink=5mm,landscape]
%\usepackage{psfrag}
%\usepackage[usenames,dvipsnames]{xcolor}
\usepackage{blkarray}
\usepackage{braket}
\usepackage{tikz}
\usepackage{tikz-3dplot}
%\usetikzlibrary{tikz-3dplot}
\usetikzlibrary{graphs}
\usetikzlibrary{datavisualization}
\usetikzlibrary{datavisualization.formats.functions}
\usepackage{pgfplotstable}
\usepgfplotslibrary{patchplots}

\setbeamercovered{transparent}

\usetheme{Pittsburgh}
%\usetheme{default}

\setbeamertemplate{sidebar right}{}
\setbeamertemplate{footline}[frame number]
%\usefonttheme{professionalfonts}

%\usepackage{sansmathaccent}
%\usepackage{bm}

%\usepackage{unicode-math}
%%\setmainfont[SlantedFont={Latin Modern Roman Slanted},SlantedFeatures={Color=000000},
%%  SmallCapsFont={TeX Gyre Termes},SmallCapsFeatures={Letters=SmallCaps}]{XITS}
%\setmathfont[math-style=ISO,sans-style=upright]{XITS Math}
%\setmathfont[range={\mathcal,\mathbfcal}]{Latin Modern Math}

\usepackage{sfmath}

%\mathversion{sans}

\newcommand{\Tr}{\mathsf{Tr}}

\definecolor{redorange}{rgb}{1.0, .25, .25}
\definecolor{citation}{rgb}{.1, 0.8, .35}
\newcommand\emm[1]{\textcolor{redorange}{{#1}}}
\newcommand\numc[1]{\textcolor{citation}{{\bf #1}}}

%\newcommand\bm[1]{{\mbox{\boldmath $#1$}}}
\newcommand\bm[1]{{\mathbf{#1}}}
%\newcommand\bm[1]{{\bf #1}}
%\newcommand\bm[1]{\ensuremath{\boldsymbol{#1}}}
%\newcommand\bm[1]{{\textbf{\it #1}}}

\title{Joint system}
\author{Ryuhei Mori}
%\institute{$\vcenter{\hbox{\includegraphics[width=30pt]{ELC_logo}}}$ Postdoctoral Fellow of ELC\\ $\vcenter{\hbox{\includegraphics[width=20pt]{titech_logo}}}$ Tokyo Institute of Technology}
\institute{Tokyo Institute of Technology}
%\date{21, Feb, 2019}



\begin{document}
\begin{frame}[plain]
\maketitle
\end{frame}


\begin{frame}{Joint system}
\begin{itemize}
\setlength{\itemsep}{2em}
\item System = Set of states \& set of measurements
\item Joint system = ``Product'' of systems.
\item Joint system of a system of a coin (two-dimentional classical system) and a system of a dice (six-dimentional classical system) is twelve-dimentional classical system.
\item What is a joint system of quantum systems ?
\end{itemize}
\end{frame}

\begin{frame}{Tensor product of linear spaces}
For linear product spaces $V$ and $W$ over a field $F$ (usually $\mathbb{R}$ or $\mathbb{C}$),
a tensor space $V\otimes W$ is a linear space spanned by $v\otimes w$ for all $v\in V,\, w\in W$.

\vspace{1em}
\begin{itemize}
\setlength{\itemsep}{2em}
%\item $\forall v\in V,\, \forall w\in W,\, v\otimes w\in V\otimes W$.
\item $\forall c\in F,\,\forall v\in V,\, \forall w\in W,\, c(v\otimes w) = (cv)\otimes w = v\otimes (cw) $.
\item $\forall u,v\in V,\, \forall w\in W,\, (u+v)\otimes w = u\otimes w + v\otimes w $.
\item $\forall v\in V,\, \forall w,y\in W,\, v\otimes (w+y) = v\otimes w + v\otimes y $.
\end{itemize}

\vspace{1em}
$\dim(V\otimes W) =\dim(V)\dim(W)$.

\vspace{1em}
If $V$ and $W$ are inner product spaces, $V\otimes W$ is also a inner product space defined by
\begin{equation*}
\langle v\otimes w, u\otimes y\rangle =
\langle v, u\rangle\langle w, y\rangle.
\end{equation*}
\end{frame}

\begin{frame}{Vector representation in tensor product}
Let $V:=\mathbb{R}^n,\,W:=\mathbb{R}^m$.
\small
\begin{align*}
e_i&:=\begin{blockarray}{[c]c}0&1\\\vdots\\0&i-1\\1&i\\0&i+1\\\vdots\\0&n\end{blockarray}\in\mathbb{R}^n,&
f_j&:=\begin{blockarray}{[c]c}0&1\\\vdots\\0&j-1\\1&j\\0&j+1\\\vdots\\0&m\end{blockarray}\in\mathbb{R}^m&
\end{align*}
\begin{align*}
e_i\otimes f_j&=\begin{blockarray}{[c]c}0&(1,1)\\\vdots\\0&(i,j-1)\\1&(i,j)\\0&(i,j+1)\\\vdots\\0&(n,m)\end{blockarray}\in\mathbb{R}^n\otimes\mathbb{R}^m
\end{align*}
\end{frame}

\begin{frame}{Vector representation in tensor product}
Let $V:=\mathbb{R}^n,\,W:=\mathbb{R}^m$.
\small
\begin{align*}
e_i\otimes f_j&=\begin{blockarray}{[c]c}0&(1,1)\\\vdots\\0&(i,j-1)\\1&(i,j)\\0&(i,j+1)\\\vdots\\0&(n,m)\end{blockarray}\in\mathbb{R}^n\otimes\mathbb{R}^m
\end{align*}
\begin{align*}
v\otimes w&=\begin{blockarray}{[c]c}\vdots\\v_iw_j&(i,j)\\\vdots\end{blockarray}=\begin{blockarray}{[c]}v_1 w\\v_2 w\\\vdots\\v_nw\end{blockarray}\in\mathbb{R}^n\otimes\mathbb{R}^m
\end{align*}
\end{frame}


\begin{frame}{Linear spaces}
\begin{itemize}
\setlength{\itemsep}{2em}
\item $L(V,W)$: A linear space spanned by linear maps from a linear space $V$ to a linear space $W$.
\item $L(V) := L(V,V)$.
\item $H(V)$: A real linear space spanned by Hermitian operators acting on a complex linear space $V$.
\item $J(V,W) := L(H(V),H(W))$.
\end{itemize}
\end{frame}

\begin{frame}{Tensor product of linear maps}
\begin{equation*}
L(V, X)\otimes L(W, Y)\cong
L(V\otimes W, X\otimes Y)
\end{equation*}
since the both the linear spaces have dimension
\begin{equation*}
\dim(V)\dim(W)\dim(X)\dim(Y).
\end{equation*}
%For $A\in L(V,X),\,B\in L(W,Y)$,\\
% $A\otimes B\in L(V, X)\otimes L(W, Y)$  can be regarded as an element of $L(V\otimes W, X\otimes Y)$.
% $A\otimes B\in L(V\otimes W, X\otimes Y)$ is defined by
A natural choice of an isomorphism is 
\begin{align*}
\Phi:\,& L(V, X)\otimes L(W, Y)\to L(V\otimes W, X\otimes Y)\\
& A\otimes B \longmapsto (v\otimes w \mapsto A(v)\otimes B(w)).
\end{align*}

\begin{equation*}
A\otimes B=
\begin{bmatrix}
A_{11}B & A_{12} B &\dotsc & A_{1m}B\\
A_{21}B & A_{22} B &\dotsc & A_{2m}B\\
\vdots & \vdots & \vdots & \vdots\\
A_{n1}B & A_{n2} B &\dotsc & A_{nm}B\\
\end{bmatrix}
\end{equation*}

\end{frame}

\begin{frame}{Tensor product of Hermitian maps}
\begin{equation*}
H(V)\otimes H(W)\cong
H(V\otimes W)
\end{equation*}
since the both the linear spaces have dimension
\begin{equation*}
\dim(V)^2\dim(W)^2.
\end{equation*}
A natural choice of an isomorphism is 
\begin{align*}
\Phi:\,& H(V)\otimes H(W)\to H(V\otimes W)\\
& A\otimes B \longmapsto (v\otimes w \mapsto A(v)\otimes B(w)).
\end{align*}
\end{frame}


\begin{frame}{Joint quantum system}
A quantum system on a complex linear space $V$: A states and a measurements are elements of $H(V)$ and ...

\vspace{3em}
For a quantum systems on $V$ and $W$, 
a joint system is a quantum system on $V\otimes W$.
\end{frame}

\begin{frame}{Examples: two-qubit system}
Examples of states
\begin{itemize}
\setlength{\itemsep}{2em}
\item $\ket{0}\bra{0}\otimes \ket{1}\bra{1} = \ket{01}\bra{01}$
\item $\frac12 (\ket{0}\bra{0}\otimes \ket{0}\bra{0} + \ket{0}\bra{0}\otimes \ket{1}\bra{1}) =
 \ket{0}\bra{0}\otimes \frac12(\ket{0}\bra{0} + \ket{1}\bra{1}) = \ket{0}\bra{0}\otimes \frac12 I$.
\item $\frac12 (\ket{1}\bra{1}\otimes \ket{0}\bra{0} + \ket{0}\bra{0}\otimes \ket{1}\bra{1})$.
\item $\frac12(\ket{0}\bra{0}\otimes \ket{0}\bra{0} + \ket{0}\bra{1}\otimes \ket{0}\bra{1}  + \ket{1}\bra{0}\otimes \ket{1}\bra{0} + \ket{1}\bra{1}\otimes \ket{1}\bra{1} )
= \ket{\varphi}\bra{\varphi}$ for $\ket{\varphi}:=\frac1{\sqrt{2}}(\ket{0}\otimes\ket{0}+\ket{1}\otimes\ket{1})$.
\end{itemize}
\end{frame}

\if0
\begin{frame}{Joint system}
Linear space of joint system is regarded as a \emm{tensor product} of linear spaces for the single systems
(justified by two postulates, no-signaling and tomographic locality [Barrett 2007 \numc{275}]).
\begin{align*}
C_1 \otimes_{\min} C_2
&:=\Bigg\{\omega \in V_1\otimes V_2 \mid \omega = \sum_i \lambda_i \omega^{(1)}_i\otimes \omega^{(2)}_i,\\
&\qquad \lambda_i \ge 0,\, \omega^{(1)}_i \in C_1,\, \omega^{(2)}_i \in C_2\Bigg\}\\
C_1 \otimes_{\max} C_2
&:=(C^*_1 \otimes_{\min} C^*_2 )^*
\end{align*}

\vspace{2em}
\begin{equation*}
(C_1 \otimes_{\max} C_2)
\supseteq
C_{1,2}\supseteq
(C_1 \otimes_{\min} C_2)
\end{equation*}
\end{frame}
\fi

%\begin{frame}{Tomography}
%For given density operator $\rho\in D()$.
%\end{frame}

\begin{frame}{Local tomography}
For measurements $\{P_a\}_a$ of quantum system on $V$
and $\{Q_b\}_b$ of quantum system on $W$,
a measurement $\{P_a\otimes Q_b\}_{a,b}$ in the joint system is said to be \emm{local}.

\vspace{3em}
A useful formula.
\begin{align*}
\Tr(A\otimes B) &= \sum_{i,j}\bra{i}\otimes\bra{j}A\otimes B \ket{i}\otimes\ket{j}\\
 &= \sum_{i,j}\bra{i}A\ket{i}\bra{j}B\ket{j}\\
&= \Tr(A)\Tr(B)
\end{align*}
\end{frame}

\begin{frame}{Separable states \& entangled states}
A quantum state $\rho$ in a joint system is said to be \emm{separable} if
\begin{align*}
\rho = \sum_{i}p_i \rho_1^i \otimes \rho_2^i
\end{align*}
for some probability distribution $p$ and quantum states $\{\rho_1^i\}$ and $\{\rho_2^i\}$ for subsystems.

\vspace{2em}
If a quantum state is not separable, the state is said to be \emm{entangled} state.
\end{frame}

\begin{frame}{Partial trace and reduced density matrix}
A probability of outcome of local measurement in a joint system is
\begin{align*}
P(a, b) = \Tr(\rho (P_a \otimes Q_b)).
\end{align*}
\begin{align*}
\sum_b P(a, b) &= \sum_b \Tr(\rho (P_a \otimes Q_b))\\
 &=  \Tr\left(\rho \left(P_a \otimes \sum_b Q_b\right)\right)\\
 &=  \Tr\left(\rho \left(P_a \otimes I\right)\right)\\
 &=  \Tr(\Tr_W(\rho) P_a).
\end{align*}
\begin{equation*}
\end{equation*}
\end{frame}

\begin{frame}{Reduced state of a pure state is not necessarily pure}
A two-qubit pure state (called Bell state, Bell pair or EPR pair)
\begin{align*}
\ket{\varphi} := \frac1{\sqrt{2}}(\ket{00}+\ket{11})
\end{align*}
\begin{align*}
\ket{\varphi}\bra{\varphi}&=\frac12(\ket{0}\bra{0}\otimes \ket{0}\bra{0} + \ket{0}\bra{1}\otimes \ket{0}\bra{1} \\
&\qquad + \ket{1}\bra{0}\otimes \ket{1}\bra{0} + \ket{1}\bra{1}\otimes \ket{1}\bra{1} )
\end{align*}
By taking the partial trace for the second qubit, we obtain a reduced density matrix $I/2$.
\end{frame}

%\begin{frame}{No-cloning theorem}
%\end{frame}

\if0
\begin{frame}{Scmidt decomposition}
\begin{theorem}[Schmidt decomposition]
For any pure state $\ket{\psi}\in V\otimes W$, there exist orthonormal basis $\{\ket{v_i}\}$ of $V$ and $\{\ket{w_i}\}$ of $W$ such that
\begin{equation*}
\ket{\psi}=\sum_i\lambda_i \ket{v_i}\ket{w_i}.
\end{equation*}
\end{theorem}
\begin{proof}[Sketch of a proof]
By a natural isomorphism
\begin{align*}
\Phi:& V\otimes W \to L(W,V)\\
%& v\otimes w \mapsto (u\mapsto \langle w, u\rangle v)
&\ket{v}\ket{w}\mapsto \ket{v}\bra{w}
\end{align*}
the Schmidt decomposition for $V\otimes W$ corresponds to the singular value decomposition for $L(W, V)$.
\end{proof}
\end{frame}
\fi

\if0
\begin{frame}{Spectral decomposition of Hermitian operator}
For any $H\in H(\mathbb{C}^n)$ and $v\in \mathbb{C}^n$,
\begin{equation*}
v,\, Hv,\, H^2v,\,\dotsc,H^nv
\end{equation*}
are linearly dependent.
\begin{align*}
0 &= a_0v+a_1 Tv+\dotsb+a_nT^nv\\
 &= c(T-\lambda_0 I)(T-\lambda_1 I)\dotsm(T-\lambda_n I)v
\end{align*}
which means at least one $T-\lambda_i I$ is not injective.
\begin{equation*}
\lambda \langle v, v\rangle
=\langle \lambda v, v\rangle
=\langle H v, v\rangle
=\langle v, H v\rangle
=\langle v, \lambda v\rangle
=\lambda^* \langle v, v\rangle
\end{equation*}
\begin{equation*}
\lambda \langle v, w\rangle
=\langle \lambda v, w\rangle
\end{equation*}

\end{frame}

\begin{frame}{Purification}
\begin{theorem}
For any density matrix $\rho$ on $V$, there exists a pure state $\ket{\psi}$ of a joint system on $V\otimes W$ for some $W$ such that $\Tr_W(\ket{\psi}\bra{\psi})=\rho$.
\end{theorem}
\end{frame}
\fi

\begin{frame}{Superdense coding}
Alice can send \emm{two} bits to Bob by sending a single qubit and using a shared Bell state.
\begin{align*}
\ket{\varphi_{00}} &= \frac1{\sqrt{2}}(\ket{00}+\ket{11})\\
\ket{\varphi_{01}} &= \frac1{\sqrt{2}}(\ket{10}+\ket{01}),& \text{by $X$}\\
\ket{\varphi_{10}} &= \frac1{\sqrt{2}}(\ket{00}-\ket{11}),& \text{by $Z$}\\
\ket{\varphi_{11}} &= \frac1{\sqrt{2}}(\ket{10}-\ket{01}),& \text{by $XZ$}\\
\end{align*}
These are \emm{orthogonal}.
\end{frame}


\if0
\begin{frame}{Quantum teleportation}
\end{frame}

\begin{frame}{Assignments}
\begin{itemize}
\item Show reduced density matrices (both sides) of your three favorite quantum states (not necessarily pure state) in any joint system.
\end{itemize}
\end{frame}
\fi

\end{document}
