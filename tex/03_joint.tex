\documentclass{beamer}
%\usepackage{xspace}
\usepackage{amsmath,amssymb}
\usepackage{graphicx}
%\usepackage{svg}
%\usepackage{pgfpages}
%\pgfpagesuselayout{4 on 1}[a4paper,border shrink=5mm,landscape]
%\usepackage{psfrag}
%\usepackage[usenames,dvipsnames]{xcolor}
\usepackage{blkarray}
\usepackage{braket}
\usepackage{tikz}
\usepackage{tikz-3dplot}
%\usetikzlibrary{tikz-3dplot}
\usetikzlibrary{graphs}
\usetikzlibrary{datavisualization}
\usetikzlibrary{datavisualization.formats.functions}
\usepackage{pgfplotstable}
\usepgfplotslibrary{patchplots}

\setbeamercovered{transparent}

\usetheme{Pittsburgh}
%\usetheme{default}

\setbeamertemplate{sidebar right}{}
\setbeamertemplate{footline}[frame number]
%\usefonttheme{professionalfonts}

%\usepackage{sansmathaccent}
%\usepackage{bm}

%\usepackage{unicode-math}
%%\setmainfont[SlantedFont={Latin Modern Roman Slanted},SlantedFeatures={Color=000000},
%%  SmallCapsFont={TeX Gyre Termes},SmallCapsFeatures={Letters=SmallCaps}]{XITS}
%\setmathfont[math-style=ISO,sans-style=upright]{XITS Math}
%\setmathfont[range={\mathcal,\mathbfcal}]{Latin Modern Math}

\usepackage{sfmath}

%\mathversion{sans}

\newcommand{\Tr}{\mathsf{Tr}}

\definecolor{redorange}{rgb}{1.0, .25, .25}
\definecolor{citation}{rgb}{.1, 0.8, .35}
\newcommand\emm[1]{\textcolor{redorange}{{#1}}}
\newcommand\numc[1]{\textcolor{citation}{{\bf #1}}}

%\newcommand\bm[1]{{\mbox{\boldmath $#1$}}}
\newcommand\bm[1]{{\mathbf{#1}}}
%\newcommand\bm[1]{{\bf #1}}
%\newcommand\bm[1]{\ensuremath{\boldsymbol{#1}}}
%\newcommand\bm[1]{{\textbf{\it #1}}}

\title{Joint system}
\author{Ryuhei Mori}
%\institute{$\vcenter{\hbox{\includegraphics[width=30pt]{ELC_logo}}}$ Postdoctoral Fellow of ELC\\ $\vcenter{\hbox{\includegraphics[width=20pt]{titech_logo}}}$ Tokyo Institute of Technology}
\institute{Tokyo Institute of Technology}
%\date{21, Feb, 2019}



\begin{document}
\begin{frame}[plain]
\maketitle
\end{frame}


\begin{frame}{Joint system}
\begin{itemize}
\setlength{\itemsep}{2em}
\item System = Set of states \& set of measurements
\item Joint system = ``Product'' of systems.
\item Joint system of a system of a coin (two-dimentional classical system) and a system of a dice (six-dimentional classical system) is twelve-dimentional classical system.
\item What is a joint system of quantum systems ?
\end{itemize}
\end{frame}

\begin{frame}{Tensor product of linear spaces}
For linear product spaces $V$ and $W$ over a field $F$ (usually $\mathbb{R}$ or $\mathbb{C}$),
a tensor space $V\otimes W$ is a linear space spanned by $v\otimes w$ for all $v\in V,\, w\in W$.

\vspace{1em}
\begin{itemize}
\setlength{\itemsep}{2em}
%\item $\forall v\in V,\, \forall w\in W,\, v\otimes w\in V\otimes W$.
\item $\forall c\in F,\,\forall v\in V,\, \forall w\in W,\, c(v\otimes w) = (cv)\otimes w = v\otimes (cw) $.
\item $\forall u,v\in V,\, \forall w\in W,\, (u+v)\otimes w = u\otimes w + v\otimes w $.
\item $\forall v\in V,\, \forall w,y\in W,\, v\otimes (w+y) = v\otimes w + v\otimes y $.
\end{itemize}

\vspace{1em}
$\dim(V\otimes W) =\dim(V)\dim(W)$.

\vspace{1em}
If $V$ and $W$ are inner product spaces, $V\otimes W$ is also a inner product space defined by
\begin{equation*}
\langle v\otimes w, u\otimes y\rangle =
\langle v, u\rangle\langle w, y\rangle.
\end{equation*}
\end{frame}

\begin{frame}{Vector representation in tensor product}
Let $V:=\mathbb{R}^n,\,W:=\mathbb{R}^m$.
\small
\begin{align*}
e_i&:=\begin{blockarray}{[c]c}0&1\\\vdots\\0&i-1\\1&i\\0&i+1\\\vdots\\0&n\end{blockarray}\in\mathbb{R}^n,&
f_j&:=\begin{blockarray}{[c]c}0&1\\\vdots\\0&j-1\\1&j\\0&j+1\\\vdots\\0&m\end{blockarray}\in\mathbb{R}^m&
\end{align*}
\begin{align*}
e_i\otimes f_j&=\begin{blockarray}{[c]c}0&(1,1)\\\vdots\\0&(i,j-1)\\1&(i,j)\\0&(i,j+1)\\\vdots\\0&(n,m)\end{blockarray}\in\mathbb{R}^n\otimes\mathbb{R}^m
\end{align*}
\end{frame}

\begin{frame}{Vector representation in tensor product}
Let $V:=\mathbb{R}^n,\,W:=\mathbb{R}^m$.
\small
\begin{align*}
e_i\otimes f_j&=\begin{blockarray}{[c]c}0&(1,1)\\\vdots\\0&(i,j-1)\\1&(i,j)\\0&(i,j+1)\\\vdots\\0&(n,m)\end{blockarray}\in\mathbb{R}^n\otimes\mathbb{R}^m
\end{align*}
\begin{align*}
v\otimes w&=\begin{blockarray}{[c]c}\vdots\\v_iw_j&(i,j)\\\vdots\end{blockarray}=\begin{blockarray}{[c]}v_1 w\\v_2 w\\\vdots\\v_nw\end{blockarray}\in\mathbb{R}^n\otimes\mathbb{R}^m
\end{align*}
\end{frame}

\begin{frame}{Tensor product of linear maps}
\end{frame}

\begin{frame}{Linear spaces}
\begin{itemize}
\setlength{\itemsep}{2em}
\item $L(V,W)$
\end{itemize}
\end{frame}

\begin{frame}{Joint system}
Linear space of joint system is regarded as a \emm{tensor product} of linear spaces for the single systems
(justified by two postulates, no-signaling and tomographic locality [Barrett 2007 \numc{275}]).
\begin{align*}
C_1 \otimes_{\min} C_2
&:=\Bigg\{\omega \in V_1\otimes V_2 \mid \omega = \sum_i \lambda_i \omega^{(1)}_i\otimes \omega^{(2)}_i,\\
&\qquad \lambda_i \ge 0,\, \omega^{(1)}_i \in C_1,\, \omega^{(2)}_i \in C_2\Bigg\}\\
C_1 \otimes_{\max} C_2
&:=(C^*_1 \otimes_{\min} C^*_2 )^*
\end{align*}

\vspace{2em}
\begin{equation*}
(C_1 \otimes_{\max} C_2)
\supseteq
C_{1,2}\supseteq
(C_1 \otimes_{\min} C_2)
\end{equation*}
\end{frame}

\begin{frame}{Tomography}
\end{frame}

\begin{frame}{Local tomography}
\end{frame}

\begin{frame}{Entanglements}
\end{frame}

\begin{frame}{Partial trace and reduced density matrix}
\end{frame}

\begin{frame}{Reduced state of pure state is not nessecarily pure}
\end{frame}

\begin{frame}{No-cloning theorem}
\end{frame}

\begin{frame}{Scmidt decomposition}
\end{frame}

\begin{frame}{Purification}
\end{frame}

\begin{frame}{Superdense coding}
\end{frame}

\begin{frame}{Quantum teleportation}
\end{frame}


\end{document}
