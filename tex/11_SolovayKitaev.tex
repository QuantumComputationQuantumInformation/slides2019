\documentclass{beamer}
%\usepackage{xspace}
\usepackage{amsmath,amssymb}
\usepackage{graphicx}
%\usepackage{svg}
%\usepackage{pgfpages}
%\pgfpagesuselayout{4 on 1}[a4paper,border shrink=5mm,landscape]
%\usepackage{psfrag}
%\usepackage[usenames,dvipsnames]{xcolor}
\usepackage{braket}
\usepackage{qcircuit}
%\usepackage{algorithmicx}
\usepackage{algpseudocode}
\usepackage{tikz}
\usepackage{tikz-3dplot}
\usetikzlibrary{circuits.logic.US}
\usetikzlibrary{graphs}
\usetikzlibrary{datavisualization}
\usetikzlibrary{datavisualization.formats.functions}
\usepackage{pgfplotstable}
\usepgfplotslibrary{patchplots}

\setbeamercovered{transparent}

\usetheme{Pittsburgh}
%\usetheme{default}

\setbeamertemplate{sidebar right}{}
\setbeamertemplate{footline}[frame number]
%\usefonttheme{professionalfonts}

%\usepackage{sansmathaccent}
%\usepackage{bm}

%\usepackage{unicode-math}
%%\setmainfont[SlantedFont={Latin Modern Roman Slanted},SlantedFeatures={Color=000000},
%%  SmallCapsFont={TeX Gyre Termes},SmallCapsFeatures={Letters=SmallCaps}]{XITS}
%\setmathfont[math-style=ISO,sans-style=upright]{XITS Math}
%\setmathfont[range={\mathcal,\mathbfcal}]{Latin Modern Math}

\usepackage{sfmath}

%\mathversion{sans}

\newcommand{\Tr}{\mathsf{Tr}}

\definecolor{redorange}{rgb}{1.0, .25, .25}
\definecolor{citation}{rgb}{.1, 0.8, .35}
\newcommand\emm[1]{\textcolor{redorange}{{#1}}}
\newcommand\numc[1]{\textcolor{citation}{{\bf #1}}}

%\newcommand\bm[1]{{\mbox{\boldmath $#1$}}}
\newcommand\bm[1]{{\mathbf{#1}}}
%\newcommand\bm[1]{{\bf #1}}
%\newcommand\bm[1]{\ensuremath{\boldsymbol{#1}}}
%\newcommand\bm[1]{{\textbf{\it #1}}}

\title{Solovay--Kitaev theorem}
\author{Ryuhei Mori}
%\institute{$\vcenter{\hbox{\includegraphics[width=30pt]{ELC_logo}}}$ Postdoctoral Fellow of ELC\\ $\vcenter{\hbox{\includegraphics[width=20pt]{titech_logo}}}$ Tokyo Institute of Technology}
\institute{Tokyo Institute of Technology}
%\date{21, Feb, 2019}



\begin{document}
\begin{frame}[plain]
\maketitle
\end{frame}



\begin{frame}{Solovay--Kitaev theorem}
\begin{theorem}
Let $\{U_1,\dotsc,U_k\}$ be a dense subset of $\mathsf{SU}(2)$.
Then, any $U\in\mathsf{SU}(2)$ can be approxmiated with error $\epsilon$ by $\emm{[\log (1/\epsilon)]^c}$ multiplications of $\{U_1,\dotsc,U_k\}$.
\end{theorem}
\end{frame}

\begin{frame}{Solovay--Kitaev algorithm}
\begin{algorithmic}
\Function{Solovay--Kitaev}{$U$, $n$}
\If {$n=0$}
  \State \Return Basic approximation to $U$
\EndIf
\State $U_{n-1} \gets$ \Call{Solovay--Kitaev}{$U$, $n-1$}
\State $V, W \gets$ \Call{GC--Decompose}{$UU_{n-1}^\dagger$}
\State $V_{n-1} \gets$ \Call{Solovay--Kitaev}{$V$, $n-1$}
\State $W_{n-1} \gets$ \Call{Solovay--Kitaev}{$W$, $n-1$}
\State \Return $V_{n-1}W_{n-1}V_{n-1}^\dagger W_{n-1}^\dagger U_{n-1}$.
\EndFunction
\end{algorithmic}

\vspace{1em}
\begin{algorithmic}
\Function{GC--Decompose}{$\Delta$}
\State \Return ($V$, $W$) satisfying $VWV^\dagger W^\dagger = \Delta$ with $\|I - V\|,\,\|I-W\|\le c_{\mathsf{GC}}\sqrt{\|I-\Delta\|}$.
\EndFunction
\end{algorithmic}
\end{frame}

\begin{frame}{Analysis}
\begin{theorem}
If $\|I - V\|,\, \|I - W\|\le \epsilon$,
$\|V-\widetilde{V}\|,\,\|W-\widetilde{W}\| \le \delta$
\begin{equation*}
\|VWV^\dagger W^\dagger - \widetilde{V}\widetilde{W}\widetilde{V}^\dagger\widetilde{W}^\dagger\| \le c_{\mathsf{approx}} \delta\epsilon
\end{equation*}
\end{theorem}
\small
From this (surprising) theorem,
\begin{align*}
\ell_n &\le 5\ell_{n-1}\\
\epsilon_n &\le c_{\mathsf{approx}} \epsilon_{n-1}^{3/2}
\end{align*}
Then,
\begin{align*}
\ell_n&\le 5^n \ell_0\\
\log (1/[c_{\mathsf{approx}}^2 \epsilon_n]) &\ge (3/2)\log (1/[c_{\mathsf{approx}}^2 \epsilon_{n-1}])\\
&\ge (3/2)^n\log (1/[c_{\mathsf{approx}}^2 \epsilon_{0}])\\
&\ge (\ell_n/\ell_0)^{\frac{\log (3/2)}{\log 5}}\log (1/\epsilon_{0})
\end{align*}
if $\epsilon_0 < 1 / c_{\mathsf{approx}}^2$.
This concludes $\ell_n = O\left((\log (1/\epsilon))^{\frac{\log 5}{\log (3/2)}}\right)$.
\end{frame}

\begin{frame}{Proof}
\begin{theorem}
If $\|I - V\|,\, \|I - W\|\le \epsilon$,
$\|V-\widetilde{V}\|,\,\|W-\widetilde{W}\| \le \delta$
\begin{equation*}
\|VWV^\dagger W^\dagger - \widetilde{V}\widetilde{W}\widetilde{V}^\dagger\widetilde{W}^\dagger\| \le c_{\mathsf{approx}} \delta\epsilon
\end{equation*}
\end{theorem}
\end{frame}

\if0
\begin{frame}{Assignments (Deadline is Jan.\ 17)}
\small
\begin{enumerate}
\setlength{\itemsep}{1em}
\item Show a decomposition of
\begin{equation*}
\frac12\begin{bmatrix}
1&1&1&1\\
1&i&-1&-i\\
1&-1&1&-1\\
1&-i&-1&i
\end{bmatrix}
\end{equation*}
into a product of two-level unitary matrices.
\item Show a decomposition of two-level unitary
\begin{equation*}
\begin{bmatrix}
1&0&0&0&0&0&0&0\\
0&1&0&0&0&0&0&0\\
0&0&a&0&0&0&0&c\\
0&0&0&1&0&0&0&0\\
0&0&0&0&1&0&0&0\\
0&0&0&0&0&1&0&0\\
0&0&0&0&0&0&1&0\\
0&0&b&0&0&0&0&d\\
\end{bmatrix}
\end{equation*}
into a product of controlled-unitary gates.
\end{enumerate}
\end{frame}
\fi


\end{document}
