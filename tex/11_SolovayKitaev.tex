\documentclass{beamer}
%\usepackage{xspace}
\usepackage{amsmath,amssymb}
\usepackage{graphicx}
%\usepackage{svg}
%\usepackage{pgfpages}
%\pgfpagesuselayout{4 on 1}[a4paper,border shrink=5mm,landscape]
%\usepackage{psfrag}
%\usepackage[usenames,dvipsnames]{xcolor}
\usepackage{braket}
\usepackage{qcircuit}
%\usepackage{algorithmicx}
\usepackage{algpseudocode}
\usepackage{tikz}
\usepackage{tikz-3dplot}
\usetikzlibrary{circuits.logic.US}
\usetikzlibrary{graphs}
\usetikzlibrary{datavisualization}
\usetikzlibrary{datavisualization.formats.functions}
\usepackage{pgfplotstable}
\usepgfplotslibrary{patchplots}

\setbeamercovered{transparent}

\usetheme{Pittsburgh}
%\usetheme{default}

\setbeamertemplate{sidebar right}{}
\setbeamertemplate{footline}[frame number]
%\usefonttheme{professionalfonts}

%\usepackage{sansmathaccent}
%\usepackage{bm}

%\usepackage{unicode-math}
%%\setmainfont[SlantedFont={Latin Modern Roman Slanted},SlantedFeatures={Color=000000},
%%  SmallCapsFont={TeX Gyre Termes},SmallCapsFeatures={Letters=SmallCaps}]{XITS}
%\setmathfont[math-style=ISO,sans-style=upright]{XITS Math}
%\setmathfont[range={\mathcal,\mathbfcal}]{Latin Modern Math}

\usepackage{sfmath}

%\mathversion{sans}

\newcommand{\Tr}{\mathsf{Tr}}

\definecolor{redorange}{rgb}{1.0, .25, .25}
\definecolor{citation}{rgb}{.1, 0.8, .35}
\newcommand\emm[1]{\textcolor{redorange}{{#1}}}
\newcommand\numc[1]{\textcolor{citation}{{\bf #1}}}

%\newcommand\bm[1]{{\mbox{\boldmath $#1$}}}
\newcommand\bm[1]{{\mathbf{#1}}}
%\newcommand\bm[1]{{\bf #1}}
%\newcommand\bm[1]{\ensuremath{\boldsymbol{#1}}}
%\newcommand\bm[1]{{\textbf{\it #1}}}

\title{Solovay--Kitaev theorem}
\author{Ryuhei Mori}
%\institute{$\vcenter{\hbox{\includegraphics[width=30pt]{ELC_logo}}}$ Postdoctoral Fellow of ELC\\ $\vcenter{\hbox{\includegraphics[width=20pt]{titech_logo}}}$ Tokyo Institute of Technology}
\institute{Tokyo Institute of Technology}
%\date{21, Feb, 2019}



\begin{document}
\begin{frame}[plain]
\maketitle
\end{frame}



\begin{frame}{Solovay--Kitaev theorem}
\begin{theorem}
Let $\{U_1,\dotsc,U_k\}$ be a dense subset of $\mathsf{SU}(2)$.
Then, any $U\in\mathsf{SU}(2)$ can be approxmiated with error $\epsilon$ by $\emm{[\log (1/\epsilon)]^c}$ multiplications of $\{U_1,\dotsc,U_k\}$ for $c=\log 5 / \log (3/2) \approx 3.97$.
\end{theorem}
\end{frame}

\begin{frame}{Special unitary group}
\begin{itemize}
\setlength{\itemsep}{2em}
\item $\mathsf{U}(n) := \text{the set of $n\times n$ unitary matrices}.$
\item $\mathsf{SU}(n) := \text{the set of $n\times n$ unitary matrices $U$ with $\emm{\det(U)=1}$}.$
\item $\mathsf{U}(n)$ and $\mathsf{SU}(n)$ are groups.
\item For $U\in\mathsf{SU}(n)$ and $V\in\mathsf{U}(n)$, $VUV^\dagger\in\mathsf{SU}(n)$.
\item For $V\in\mathsf{U}(n)$ and $W\in\mathsf{U}(n)$, $VWV^\dagger W^\dagger\in\mathsf{SU}(n)$.
\end{itemize}
\end{frame}

\begin{frame}{Special unitary group and rotation}
For a real unit vector $\hat{n}=[n_X\ n_Y\ n_Z]$, let
\begin{equation*}
R_{\hat{n}}(\theta):=\cos\frac{\theta}2I -i\sin\frac{\theta}2(n_XX+n_YY+n_ZZ).
\end{equation*}
For any $U\in\mathsf{U}(2)$, there exist $\alpha,\,\theta\in\mathbb{R}$ and a real unit three-dimensional vector $\hat{n}$ such that
$U=\mathsf{e}^{i\alpha}R_{\hat{n}}(\theta)$.

\vspace{2em}
$U\in\mathsf{U}(2)$ is in $\mathsf{SU}(2)$ iff $\Tr(U)\in\mathbb{R}$ since two eigenvalues of $U\in\mathsf{SU}(2)$ are in the form $\{\mathsf{e}^{i\theta}, \mathsf{e}^{-i\theta}\}$.

\vspace{2em}
$U\in\mathsf{U}(2)$ is in $\mathsf{SU}(2)$ iff $\emm{U=R_{\hat{n}}(\theta)}$ for some $\theta\in\mathbb{R}$.
\end{frame}

\begin{frame}{Special unitary group and commutator}
\small
\begin{theorem}
For any $U\in\mathsf{SU}(2)$, there exist $V,\,W\in\mathsf{U}(2)$ such that $U = VWV^\dagger W^\dagger$.
\end{theorem}
\begin{proof}

\vspace{-2em}
\begin{align*}
&R_Z(\theta)R_X(\theta)R_Z(\theta)^\dagger R_X(\theta)^\dagger
= R_Z(\theta)R_X(\theta)R_Z(-\theta)R_X(-\theta)\\
&= R_Z(\theta)R_X(\theta)R_Z(-\theta)R_X(-\theta)\\
&= \left[\cos\frac{\theta}2 I - i\sin\frac{\theta}2 Z\right]
\left[\cos\frac{\theta}2 I - i\sin\frac{\theta}2 X\right]
\left[\cos\frac{\theta}2 I + i\sin\frac{\theta}2 Z\right]
\left[\cos\frac{\theta}2 I + i\sin\frac{\theta}2 X\right]\\
&= \left[\cos^4\frac{\theta}2+2\cos^2\frac{\theta}2\sin^2\frac{\theta}2-\sin^4\frac{\theta}2\right] I + \dotsb\\
%&= \left[\cos^2\frac{\theta}2(\cos^2\frac{\theta}2+2\sin^2\frac{\theta}2)-\sin^4\frac{\theta}2\right] I + \dotsb\\
&= \left[1-2\sin^4\frac{\theta}2\right] I + \dotsb
=R_{\widehat{n}_\theta}(\varphi)
%&= \left[\cos^2\frac{\theta}2 I - i\sin\frac{\theta}2\cos\frac{\theta}2(Z+X) - \sin^2\frac{\theta}2 ZX\right]\\
%&\cdot \left[\cos^2\frac{\theta}2 I + i\sin\frac{\theta}2\cos\frac{\theta}2(Z+X) - \sin^2\frac{\theta}2 ZX\right]
%&= \left[\cos^2\frac{\theta}2 I - \sin^2\frac{\theta}2 ZX\right]^2 - i\sin\frac{\theta}2\cos\frac{\theta}2(Z+X) \right]\\
\end{align*}
$\cos\frac{\varphi}2 = 1-2\sin^4\frac{\theta}2$.
%$\sin^2\frac{\varphi}2 = 4\sin^4\frac{\theta}2 - 4\sin^8\frac{\theta}2$.
%$\sin\frac{\varphi}2 = 2\sin^2\frac{\theta}2\sqrt{1 - \sin^4\frac{\theta}2}$.
For some $S\in \mathsf{U}(2)$ and $\varphi\in\mathbb{R}$, $U=SR_{\widehat{n}_\theta}(\varphi)S^\dagger$.
For $V:= SR_Z(\theta)S^\dagger$ and $W:= SR_X(\theta)S^\dagger$,
$U=VWV^\dagger W^\dagger$.
\end{proof}
\end{frame}

\begin{frame}{Rotation matrix and distance}
\begin{align*}
\|I - R_{\widehat{n}}(\theta)\|
&= \left\|\begin{bmatrix}1-\mathsf{e}^{i\theta/2}&0\\0&1-\mathsf{e}^{-i\theta/2}\end{bmatrix}\right\|\\
&= \left|1-\mathsf{e}^{i\theta/2}\right|\\
&= 2\left|\sin\frac{\theta}4\right|
\end{align*}
For $U\in \mathsf{SU}(2)$, $V,\,W\in\mathsf{SU}(2)$ satisfying $U=VWV^\dagger W^\dagger$ in the construction
\begin{equation*}
\|I - U\| = 2\left|\sin\frac{\varphi}4\right|
= 2\sqrt{\frac{1-\cos\frac{\varphi}2}2}
= 2 \sin^2\frac{\theta}2
\approx 8 \sin^2\frac{\theta}4
= 2\|I - V\|^2
\end{equation*}
\end{frame}

\begin{frame}{Solovay--Kitaev algorithm}
\begin{algorithmic}
\Function{Solovay--Kitaev}{$U$, $n$}
\If {$n=0$}
  \State \Return Basic approximation to $U$
\EndIf
\State $U_{n-1} \gets$ \Call{Solovay--Kitaev}{$U$, $n-1$}
\State $V, W \gets$ \Call{GC--Decompose}{$UU_{n-1}^\dagger$}
\State $V_{n-1} \gets$ \Call{Solovay--Kitaev}{$V$, $n-1$}
\State $W_{n-1} \gets$ \Call{Solovay--Kitaev}{$W$, $n-1$}
\State \Return $V_{n-1}W_{n-1}V_{n-1}^\dagger W_{n-1}^\dagger U_{n-1}$.
\EndFunction
\end{algorithmic}

\vspace{1em}
\begin{algorithmic}
\Function{GC--Decompose}{$\Delta$}
\State \Return ($V$, $W$) satisfying $VWV^\dagger W^\dagger = \Delta$ with $\|I - V\|,\,\|I-W\|\le c_{\mathsf{GC}}\sqrt{\|I-\Delta\|}$.
\EndFunction
\end{algorithmic}
\end{frame}

\begin{frame}{Analysis}
\small
\begin{theorem}
If $\|I - V\|,\, \|I - W\|\le \delta$,
$\|V-\widetilde{V}\|,\,\|W-\widetilde{W}\| \le \Delta$
\begin{equation*}
\|VWV^\dagger W^\dagger - \widetilde{V}\widetilde{W}\widetilde{V}^\dagger\widetilde{W}^\dagger\| \le c_{\mathsf{approx}} \Delta(\delta + \Delta)
\end{equation*}
\end{theorem}
From this (surprising) theorem for $\Delta=\epsilon_{n-1}$, $\delta=c_{\mathsf{GC}}\sqrt{\epsilon_{n-1}}$,
\begin{align*}
\ell_n &\le 5\ell_{n-1}\\
\epsilon_n &\le c_{\mathsf{approx}} \epsilon_{n-1}^{3/2}
\end{align*}
Then,
\begin{align*}
\ell_n&\le 5^n \ell_0\\
c_{\mathsf{approx}}^2\epsilon_n &\le c_{\mathsf{approx}}^3 \epsilon_{n-1}^{3/2}=(c_{\mathsf{approx}}^2\epsilon_{n-1})^{3/2} \\
&\le (c_{\mathsf{approx}}^2\epsilon_{0})^{(3/2)^n} 
%\log (1/[c_{\mathsf{approx}}^2 \epsilon_n]) &\ge (3/2)\log (1/[c_{\mathsf{approx}}^2 \epsilon_{n-1}])\\
%&\ge (3/2)^n\log (1/[c_{\mathsf{approx}}^2 \epsilon_{0}])\\
%&\ge (\ell_n/\ell_0)^{\frac{\log (3/2)}{\log 5}}\log (1/\epsilon_{0})
\end{align*}
If $\epsilon_0 < 1 / c_{\mathsf{approx}}^2$, $\ell_n = O\left((\log (1/\epsilon))^{\frac{\log 5}{\log (3/2)}}\right)$.
\end{frame}

\begin{frame}{Proof}
\small
\begin{theorem}
If $\|I - V\|,\, \|I - W\|\le \delta$,
$\|V-\widetilde{V}\|,\,\|W-\widetilde{W}\| \le \Delta$
\begin{equation*}
\|VWV^\dagger W^\dagger - \widetilde{V}\widetilde{W}\widetilde{V}^\dagger\widetilde{W}^\dagger\| \le c_{\mathsf{approx}} \Delta\delta
\end{equation*}
\end{theorem}
\begin{proof}
\begin{align*}
 \widetilde{V}\widetilde{W}\widetilde{V}^\dagger\widetilde{W}^\dagger
&= VWV^\dagger W^\dagger + 
\Delta_VWV^\dagger W^\dagger + 
V\Delta_WV^\dagger W^\dagger\\
&\quad + VW\Delta_V^\dagger W^\dagger + VWV^\dagger\Delta_W^\dagger + O(\Delta^2)
\end{align*}
\begin{align*}
\|VWV^\dagger W^\dagger - \widetilde{V}\widetilde{W}\widetilde{V}^\dagger\widetilde{W}^\dagger\|
&\le
\|\Delta_VWV^\dagger W^\dagger + VW\Delta_V^\dagger W^\dagger\|\\
&\quad+\|V\Delta_WV^\dagger W^\dagger + VWV^\dagger\Delta_W^\dagger\| + O(\Delta^2)
\end{align*}
\begin{align*}
\Delta_VWV^\dagger W^\dagger + VW\Delta_V^\dagger W^\dagger
=
\Delta_VV^\dagger + V\Delta_V^\dagger +
\Delta_V\delta_W V^\dagger + V\Delta_V^\dagger \delta_W^\dagger
\end{align*}
\end{proof}
\end{frame}

\begin{frame}{Commutator and controlled-unitary}
\begin{theorem}
For any $U\in\mathsf{SU}(2)$, controlled-$U$ gate with $n$ controlled qubits can be realized by $O(n^2)$ CNOT and arbitrary single-qubit gates without ancillas (working qubits).
\end{theorem}
\begin{proof}
\small
Induction on $n$.
For the group commutator decomposition $U=VWV^\dagger W^\dagger$ using $V,\,W\in \mathsf{SU}(2)$,
\[
\Qcircuit @C=2em @R=1em {
& \ctrl{6} & \qw      & \ctrl{6} & \qw     & \qw\\
& \ctrl{5} & \qw      & \ctrl{5} & \qw     & \qw\\
& \ctrl{4} & \qw      & \ctrl{4} & \qw     & \qw\\
& \qw      & \ctrl{3} & \qw      & \ctrl{3}& \qw\\
& \qw      & \ctrl{2} & \qw      & \ctrl{2}& \qw\\
& \qw      & \ctrl{1} & \qw      & \ctrl{1}& \qw\\
& \gate{W^\dagger} & \gate{V^\dagger} & \gate{W} & \gate{V}& \qw
}
\]
$S_n = 4 S_{n/2} = 4^{\log n}S_1 = O(n^2)$.
\end{proof}
\end{frame}

\begin{frame}{Controlled-unitary for $U(2)$}
\end{frame}

\if0
\begin{frame}{Assignments (Deadline is Jan.\ 17)}
\small
\begin{enumerate}
\setlength{\itemsep}{1em}
\item Show a decomposition of
\begin{equation*}
\frac12\begin{bmatrix}
1&1&1&1\\
1&i&-1&-i\\
1&-1&1&-1\\
1&-i&-1&i
\end{bmatrix}
\end{equation*}
into a product of two-level unitary matrices.
\item Show a decomposition of two-level unitary
\begin{equation*}
\begin{bmatrix}
1&0&0&0&0&0&0&0\\
0&1&0&0&0&0&0&0\\
0&0&a&0&0&0&0&c\\
0&0&0&1&0&0&0&0\\
0&0&0&0&1&0&0&0\\
0&0&0&0&0&1&0&0\\
0&0&0&0&0&0&1&0\\
0&0&b&0&0&0&0&d\\
\end{bmatrix}
\end{equation*}
into a product of controlled-unitary gates.
\end{enumerate}
\end{frame}
\fi


\end{document}
