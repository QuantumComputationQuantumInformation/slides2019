\documentclass{beamer}
%\usepackage{xspace}
\usepackage{amsmath,amssymb}
\usepackage{graphicx}
%\usepackage{svg}
%\usepackage{pgfpages}
%\pgfpagesuselayout{4 on 1}[a4paper,border shrink=5mm,landscape]
%\usepackage{psfrag}
%\usepackage[usenames,dvipsnames]{xcolor}
\usepackage{braket}
\usepackage{qcircuit}
%\usepackage{algorithmicx}
\usepackage{algpseudocode}
\usepackage{tikz}
\usepackage{tikz-3dplot}
\usetikzlibrary{circuits.logic.US}
\usetikzlibrary{graphs}
\usetikzlibrary{datavisualization}
\usetikzlibrary{datavisualization.formats.functions}
\usepackage{pgfplotstable}
\usepgfplotslibrary{patchplots}

\setbeamercovered{transparent}

\usetheme{Pittsburgh}
%\usetheme{default}

\setbeamertemplate{sidebar right}{}
\setbeamertemplate{footline}[frame number]
%\usefonttheme{professionalfonts}

%\usepackage{sansmathaccent}
%\usepackage{bm}

%\usepackage{unicode-math}
%%\setmainfont[SlantedFont={Latin Modern Roman Slanted},SlantedFeatures={Color=000000},
%%  SmallCapsFont={TeX Gyre Termes},SmallCapsFeatures={Letters=SmallCaps}]{XITS}
%\setmathfont[math-style=ISO,sans-style=upright]{XITS Math}
%\setmathfont[range={\mathcal,\mathbfcal}]{Latin Modern Math}

\usepackage{sfmath}

%\mathversion{sans}

\newcommand{\Tr}{\mathsf{Tr}}

\definecolor{redorange}{rgb}{1.0, .25, .25}
\definecolor{citation}{rgb}{.1, 0.8, .35}
\newcommand\emm[1]{\textcolor{redorange}{{#1}}}
\newcommand\numc[1]{\textcolor{citation}{{\bf #1}}}

%\newcommand\bm[1]{{\mbox{\boldmath $#1$}}}
\newcommand\bm[1]{{\mathbf{#1}}}
%\newcommand\bm[1]{{\bf #1}}
%\newcommand\bm[1]{\ensuremath{\boldsymbol{#1}}}
%\newcommand\bm[1]{{\textbf{\it #1}}}

\title{Grover's algorithm}
\author{Ryuhei Mori}
%\institute{$\vcenter{\hbox{\includegraphics[width=30pt]{ELC_logo}}}$ Postdoctoral Fellow of ELC\\ $\vcenter{\hbox{\includegraphics[width=20pt]{titech_logo}}}$ Tokyo Institute of Technology}
\institute{Tokyo Institute of Technology}
%\date{21, Feb, 2019}



\begin{document}
\begin{frame}[plain]
\maketitle
\end{frame}


\begin{frame}{Searching problem}
Searching problem:
\begin{equation*}
f: \{1,2,\dotsc,N\}\to\{0,1\}
\end{equation*}
Find $x\in\{1,2,\dotsc,N\}$ satisfying $f(x) = 1$.

\vspace{3em}
\centering
\Large
How many times, do we have to evaluate $f(x)$ ?

\vspace{2em}
\large
Obviously, $O(N)$.
%
%\vspace{2em}
%Grover's algorithm can do this with $\emm{O(\sqrt{N})}$ evaluations.
\end{frame}

\begin{frame}{Quantum searching problem}
Unitary oracle
\begin{equation*}
U_f \ket{x}\ket{y} = \ket{x}\ket{y\oplus f(x)}.
\end{equation*}
Find $x\in\{1,2,\dotsc,N\}$ satisfying $f(x) = 1$.

\vspace{3em}
\centering
\Large
How many times, do we have to evaluate $U_f$ ?

\vspace{2em}
\large
$\emm{O(\sqrt{N})}$ by Grover's algorithm.
\end{frame}

\begin{frame}{Quantum searching problem}
Another unitary
\begin{equation*}
V_f \ket{x} = (-1)^{f(x)}\ket{x}.
\end{equation*}

\vspace{2em}
\[
\Qcircuit @C=2em @R=1em {
\lstick{\ket{x_1}}   & \multigate{3}{U_f}&  \qw\\
\lstick{\ket{x_2}}   & \ghost{U_f} & \qw\\
\lstick{\ket{x_3}}   & \ghost{U_f} & \qw\\
\lstick{\ket{-}}     & \ghost{U_f} & \qw
}
\]


\end{frame}

\end{document}
